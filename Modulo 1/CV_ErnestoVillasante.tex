%%%%%%%%%%%%%%%%%
% This is an sample CV template created using altacv.cls
% (v1.6.4, 13 Nov 2021) written by LianTze Lim (liantze@gmail.com). Now compiles with pdfLaTeX, XeLaTeX and LuaLaTeX.
%
%% It may be distributed and/or modified under the
%% conditions of the LaTeX Project Public License, either version 1.3
%% of this license or (at your option) any later version.
%% The latest version of this license is in
%%    http://www.latex-project.org/lppl.txt
%% and version 1.3 or later is part of all distributions of LaTeX
%% version 2003/12/01 or later.
%%%%%%%%%%%%%%%%

%% Use the "normalphoto" option if you want a normal photo instead of cropped to a circle
% \documentclass[10pt,a4paper,normalphoto]{altacv}

\documentclass[10pt,a4paper,ragged2e,withhyper]{altacv}
%% AltaCV uses the fontawesome5 and packages.
%% See http://texdoc.net/pkg/fontawesome5 for full list of symbols.

% Change the page layout if you need to
\geometry{left=1.25cm,right=1.25cm,top=1.5cm,bottom=1.5cm,columnsep=1.2cm}

% The paracol package lets you typeset columns of text in parallel
\usepackage{paracol}

% Change the font if you want to, depending on whether
% you're using pdflatex or xelatex/lualatex
\ifxetexorluatex
  % If using xelatex or lualatex:
  \setmainfont{Roboto Slab}
  \setsansfont{Lato}
  \renewcommand{\familydefault}{\sfdefault}
\else
  % If using pdflatex:
  \usepackage[rm]{roboto}
  \usepackage[defaultsans]{lato}
  % \usepackage{sourcesanspro}
  \renewcommand{\familydefault}{\sfdefault}
\fi

% Change the colours if you want to
\definecolor{SlateGrey}{HTML}{2E2E2E}
\definecolor{LightGrey}{HTML}{666666}
\definecolor{DarkPastelRed}{HTML}{450808}
\definecolor{PastelRed}{HTML}{8F0D0D}
\definecolor{GoldenEarth}{HTML}{E7D192}
\colorlet{name}{black}
\colorlet{tagline}{PastelRed}
\colorlet{heading}{DarkPastelRed}
\colorlet{headingrule}{GoldenEarth}
\colorlet{subheading}{PastelRed}
\colorlet{accent}{PastelRed}
\colorlet{emphasis}{SlateGrey}
\colorlet{body}{LightGrey}

% Change some fonts, if necessary
\renewcommand{\namefont}{\Huge\rmfamily\bfseries}
\renewcommand{\personalinfofont}{\footnotesize}
\renewcommand{\cvsectionfont}{\LARGE\rmfamily\bfseries}
\renewcommand{\cvsubsectionfont}{\large\bfseries}


% Change the bullets for itemize and rating marker
% for \cvskill if you want to
\renewcommand{\itemmarker}{{\small\textbullet}}
\renewcommand{\ratingmarker}{\faCircle}

%% Use (and optionally edit if necessary) this .tex if you
%% want to use an author-year reference style like APA(6)
%% for your publication list
\input{pubs-authoryear}

%% Use (and optionally edit if necessary) this .tex if you
%% want an originally numerical reference style like IEEE
%% for your publication list
% \input{pubs-num}

%% sample.bib contains your publications
\addbibresource{sample.bib}

\begin{document}
\name{ernesto villasante esposito}
\tagline{Estudiante de Ingenieria Industrial}
%% You can add multiple photos on the left or right
\photoR{2.8cm}{Globe_High}
% \photoL{2.5cm}{Yacht_High,Suitcase_High}

\personalinfo{%
  % Not all of these are required!
  \email{ernestovillasante@gmail.com}
  \phone{(+54) 9 2634383719}
  \mailaddress{Salta,San Martin}
  \location{San Martin,Mendoza}
  \twitter{@ernesvillasante}
  %% You can add your own arbitrary detail with
  %% \printinfo{symbol}{detail}[optional hyperlink prefix]
  % \printinfo{\faPaw}{Hey ho!}[https://example.com/]
  %% Or you can declare your own field with
  %% \NewInfoFiled{fieldname}{symbol}[optional hyperlink prefix] and use it:
  % \NewInfoField{gitlab}{\faGitlab}[https://gitlab.com/]
  % \gitlab{your_id}
  %%
  %% For services and platforms like Mastodon where there isn't a
  %% straightforward relation between the user ID/nickname and the hyperlink,
  %% you can use \printinfo directly e.g.
  % \printinfo{\faMastodon}{@username@instace}[https://instance.url/@username]
  %% But if you absolutely want to create new dedicated info fields for
  %% such platforms, then use \NewInfoField* with a star:
  % \NewInfoField*{mastodon}{\faMastodon}
  %% then you can use \mastodon, with TWO arguments where the 2nd argument is
  %% the full hyperlink.
  % \mastodon{@username@instance}{https://instance.url/@username}
}

\makecvheader
%% Depending on your tastes, you may want to make fonts of itemize environments slightly smaller
% \AtBeginEnvironment{itemize}{\small}

%% Set the left/right column width ratio to 6:4.
\columnratio{0.6}

% Start a 2-column paracol. Both the left and right columns will automatically
% break across pages if things get too long.
\begin{paracol}{2}
\cvsection{Experiencia}

\cvevent{Encargado en taller automotriz}{GB carstyle }{Diciembre 2019-- Marzo 2020 }{San Martin, Mendoza}
\begin{itemize}
\item Experiencia con reposicion de insumos y respuestos.
\item Realización de presupuestos.
\end{itemize}

\divider

\cvevent{Encargado de empaque}{Agrícola Villasante S.A.}{Diciembre 2020-- Marzo 2021}{}
\begin{itemize}
\item Desarrollo actividades en un empaque de frutas. C
\item Despachante de camiones realizando tramites sanitarios y de aduana. \item Resepcionista de materia prima e insumos. 
\end{itemize}

\divider

\cvevent{Administrativo }{Agrícola Villasante S.A.}{
Diciembre 2021-- Marzo 2022}{San Martin, Mendoza}
\begin{itemize}
\item Realización de tramites bancarios, seguimiento y revision de cuentas bancarias. 
\item Compra de insumos, materiales  y contratación de servicios. 
\end{itemize}

\divider

\cvevent{Vendedor de frutas }{Freelancer}{Diciembre 2018-- Presente}{San Martin, Mendoza}
\begin{itemize}
\item Comercialización de frutas frescas y secas
\item Productos como ciruela, uva, nueces, almendras y pistachos.
\item Realización de trabajos acopiando grandes cantidades para industrias como secaderos de ciruela o también vendiendo al publico como comercio minorista. 
\end{itemize}


\medskip

\cvsection{Un dia en mi vida}

% Adapted from @Jake's answer from http://tex.stackexchange.com/a/82729/226
% \wheelchart{outer radius}{inner radius}{
% comma-separated list of value/text width/color/detail}
\wheelchart{1.5cm}{0.5cm}{%
  6/8em/accent!30/{Dormir},
  3/8em/accent!40/Hobbies y nuevos proyectos,
  8/8em/accent!60/Facultad,
  2/10em/accent/Gimnasio,
  5/6em/accent!20/ Familia y amigos
}

% use ONLY \newpage if you want to force a page break for
% ONLY the current column
\newpage



%% Switch to the right column. This will now automatically move to the second
%% page if the content is too long.
\switchcolumn

\cvsection{Filosofia de vida}

\begin{quote}
``Si todo parece bajo control, entonces no estás yendo lo suficiente rapido.''
\end{quote}


\cvsection{Idiomas}

\cvskill{Ingles}{4}
\divider

\cvskill{Español}{5}
\divider



%% Yeah I didn't spend too much time making all the
%% spacing consistent... sorry. Use \smallskip, \medskip,
%% \bigskip, \vspace etc to make adjustments.
\medskip

\cvsection{Educacion}

\cvevent{Ingeniería Industrial}{Universidad Nacional de Cuyo}{2019 -- 2024}{}


\divider

\cvevent{Técnico agropecuario especializado en enologia }{Escuela vitivinicola Don bosco}{2013 -- 2018}{}

\divider

\cvevent{Ingles }{Links Ingles Institute}{2017 -- 2019}{}

% \divider
\cvsection{Cursos}


\divider
\cvevent{Mecanica Automotriz }{Terciario Don Bosco}{2017}{}


\divider

\cvevent{Finanzas para empresas }{Banco Macro}{2018}{}





\end{paracol}


\end{document}
