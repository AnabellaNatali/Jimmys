\documentclass{article}
\usepackage[utf8]{inputenc}
\usepackage{paracol}
\usepackage{hyperref}
\usepackage{imakeidx}
\usepackage[T1]{fontenc}
\usepackage[backend=biber,style=ieee,sorting=ydnt]{biblatex}
\usepackage[document]{ragged2e}

\title{Infraestructuras Críticas: Sector Financiero}
\author{Anabella Bottasso \and Manuel Ignacio Lopez \and
Ernesto Villasante \and Valentin Adelaide \and Sebastian Zaragoza \thanks{ Con agradecimiento a los integrantes de la cátedra de Técnicas y Herramientas Modernas: \\ Doctor Gustavo A. Masera, Ing. Alejandro C. Gomez y Doctor Ricardo R. Palma }
}

\date{May 2022}


\makeindex



\begin{document}


\maketitle


\Centering
\textbf{
\href{https://www.uncuyo.edu.ar/}{Universidad Nacional de Cuyo} 
 }
\\

\textbf{
\href{https://ingenieria.uncuyo.edu.ar/}{Facultad de Ingeniería}  
}


\textbf{
\href{https://sites.google.com/view/tyhm/inicio}{Cátedra Técnicas y Herramientas Modernas} 
\\}
\newpage
\justify
\tableofcontents
\newpage
\begin{abstract}

    En este informe se tratará sobre el sector financiero (Sistema Financiero y Tributario, o sea entidades bancarias, información, valores e inversiones) como infraestructura crítica, las diferentes razones por las cuales lo es, y las consecuencias que conlleva un colapso en esta estructura.
\end{abstract}
\section{Motivación}
El incesante cambio del mundo nos obliga como sociedad a estar listos, ya que ante cualquier cambio brusco, los más débiles, a quienes más debemos proteger como sociedad, son los que se ven mas afectados. No obstante todos nos vemos afectados. 
\newline
\newline
Hay muchas formas de abordar esto, una de ellas es diferenciar distintos tipos de amenazas, y discriminar por los sectores que estas afectan. 
\newline
\newline
Esto permite proteger a estos sectores de una forma más efectiva, y dado el caso, elegir cuál se debe priorizar.
\newline
\newline
Además, nos permite analizar estos sectores, identificar relaciones de dependencia  unilaterales y bilaterales, estructurar óptimamente al sector y fortificar acordemente.  

\section{¿Qué es una infraestructura crítica?}
Las infraestructuras críticas son todos aquellos sistemas físicos o virtuales que facilitan funciones y servicios esenciales para apoyar a los sistemas más básicos a nivel social, económico, medioambiental y político.
\newline
\newline
Lo más comúnmente asociado con el término son las instalaciones para:
\newline
\newline
• Refugio; Calefacción (por ejemplo, gas natural, fuel oil, calefacción urbana);
\newline
\newline
• Agricultura, producción y distribución de alimentos;
\newline
\newline
• Estadísticas de educación, desarrollo de habilidades y transferencia de tecnología / subsistencia básica y tasa de desempleo;
\newline
\newline
• Abastecimiento de agua (agua potable, aguas residuales/aguas residuales, derivación de aguas superficiales (por ejemplo, diques y esclusas));
\newline
\newline
• Salud pública (hospitales, ambulancias);
\newline
\newline
• Sistemas de transporte (suministro de combustible, red ferroviaria, aeropuertos, puertos, navegación interior);
Servicios de seguridad (policía, militares).
\newline
\newline
• Generación, transmisión y distribución de electricidad; (por ejemplo, gas natural, fuel oil, carbón, energía nuclear)
\newline
\newline
• Energía renovable, que se repone naturalmente en una escala de tiempo humana, como la luz solar, el viento, la lluvia, las mareas, las olas y el calor geotérmico.
\newline
\newline
• Telecomunicación; coordinación para operaciones exitosas
\newline
\newline
• Sector económico; Bienes y servicios y servicios financieros (banca, compensación);

\section{¿Por qué el sector financiero es una infraestructura crítica?}

\href{https://www.ccn-cert.cni.es/publico/InfraestructurasCriticaspublico/DirectivaEuropea2008-114-CE.pdf}{ Según la Directiva europea 2008/114/CE del 8 de diciembre de 2008, }
\newline
\newline
“El elemento, sistema o parte de este situado en los estados miembros que es esencial para el mantenimiento de funciones sociales vitales, la salud, la integridad física, la seguridad, y el bienestar social y económico de la población, cuya perturbación o destrucción afectaría gravemente a un Estado miembro al no poder mantener esas funciones”.
\newline
\newline
El sistema financiero es una estructura crítica porque de este dependen casi todos los servicios, la comida, la población, y el mismo estado en muchos casos.
 Es una pieza clave en el funcionamiento de las economías modernas pues permite a los agentes económicos utilizar su renta futura para financiar consumo o inversión presente.
Así, el temor acerca de una devaluación cambiaria espantará a inversores resultando también en residentes que deshacen sus posiciones en los bancos del país en riesgo. Este tipo de contagio es uno de los factores que ayudan a entender la viciosa interacción entre la salud financiera de un país y su sistema financiero. El empeoramiento de la salud económica de un país o de la viabilidad de sus instituciones financieras puede retroalimentarse contribuyendo a incrementar el riesgo de parálisis de su sistema financiero y la solvencia de su emisor soberano.
El menor acceso a un crédito cada vez más caro tiene como consecuencia final la reducción del consumo y de la inversión, la desaceleración económica y un aumento del desempleo; la caída del sistema financiero implicaría desempleo total, quiebre de las mayores corporaciones y un desbalance económico, cambiario, social y político.  
En conclusión la parálisis del sistema financiero tiene un efecto muy negativo y directo sobre el crecimiento económico, el empleo y el bienestar de los ciudadanos. 

\section{Amenazas y vulnerabilidades de las infraestructuras críticas}
El Sector de Servicios Financieros representa un componente vital de la infraestructura crítica de nuestra nación. Los cortes de energía a gran escala, los desastres naturales recientes y un aumento en el número y la sofisticación de los ataques cibernéticos demuestran la amplia gama de riesgos potenciales que enfrenta el sector.
Las amenazas que pueden poner en riesgo los intereses vitales y estratégicos se han visto incrementadas en los últimos años. Tanto es así que los funcionarios de seguridad y del gobierno están preocupados por las amenazas y las vulnerabilidades a las que están expuestas las infraestructuras críticas:
\newline
\newline
1. Terrorismo
\newline
\newline
2. Crimen organizado – Es una amenaza de carácter transnacional, flexible y opaca. Tiene una gran capacidad desestabilizadora, cuyo fin es el ánimo de lucro, pero debilitando el Estado y minando la buena gobernanza económica. Una parte del Crimen Organizado es el llevado a cabo por los Grupos Violentos son los responsables de gran parte de las conductas violentas en las grandes ciudades. 
\newline
\newline
3. Proliferación de armas de destrucción masiva
\newline
\newline
4. Espionaje – Es una amenaza de primer orden para la seguridad tanto por el espionaje de otros países como por el realizado por empresas extranjeras. La inteligencia en el ciberespacio adopta el nombre de ciberinteligencia siendo su objetivo obtener cantidades ingentes de información y datos confidenciales entre los que puedan estar los de infraestructuras críticas.
\newline
\newline
5. Vulnerabilidad del ciberespacio – Las amenazas en el ciberespacio han adquirido una dimensión global que va mucho más allá de la tecnología. El objetivo es conseguir diferentes propósitos como, por ejemplo, la expansión de determinados intereses geopolíticos por parte de Estados, organizaciones terroristas y actores individuales. 
\newline
\newline
6.  Causas naturales – El impacto de las catástrofes perjudica la vida de las personas, así como a los bienes patrimoniales, al medio ambiente y al desarrollo económico. Por otro lado, las epidemias y las pandemias han aumentado su número y situaciones de riesgo. Y, finalmente, los efectos derivados del cambio climático tienen graves consecuencias. Por ejemplo, la subida de las temperaturas afecta a los niveles del mar, a la degradación del suelo y a la acidificación del océano, entre otros. 
\newline
\newline
7. Pandemias (como el Covid-19), ponen en jaque la salud de la sociedad y está conllevando múltiples cambios en nuestras rutinas, en los modelos empresariales y grandes efectos en las variables macroeconómicas, entre otros aspectos. Las empresas están reaccionando ante la crisis como pueden, respondiendo con la máxima agilidad posible y adaptándose a las nuevas demandas. Esto puede provocar que, en ocasiones, su ciberseguridad se vea afectada al pasar a un segundo plano, algo de lo que se están aprovechando los cibercriminales para realizar actividades ilícitas.
\newline
\newline
Para proteger al sistema de esto se diseñó...

\section{El Plan Sectorial Específico del sistema financiero}
El \href{https://www.cisa.gov/publication/nipp-ssp-financial-services-2015}{\textit{Plan Específico del Sector de Servicios Financieros}} detalla cómo se implementa el marco de gestión de riesgos del Plan Nacional de Protección de la Infraestructura dentro del contexto de las características únicas y el panorama de riesgos del sector. Cada Agencia de Gestión de Riesgos Sectoriales desarrolla un plan específico del sector a través de un esfuerzo coordinado que involucra a sus socios del sector público y privado. Se designa al Departamento del Tesoro como Agencia de Administración de Riesgos Sectoriales para el Sector de Servicios Financieros.
 \newline
Según el plan específico del sector de servicios financieros de CISA: 
\newline
“La seguridad y resiliencia del Sector de Servicios Financieros depende de la estrecha colaboración entre un amplio conjunto de socios, incluidas las empresas del Sector de Servicios Financieros; asociaciones gremiales sectoriales; Agencias del gobierno federal; reguladores financieros; Gobiernos estatales, locales, tribales y territoriales; y otros socios gubernamentales y del sector privado en los EEUU. y en todo el mundo. Estos socios buscan reducir los riesgos físicos y de seguridad cibernética que toman muchas formas pero, particularmente en el caso de las amenazas de seguridad cibernética, se están volviendo cada vez más apremiantes..”
\newline
\newline
Para ello se lograron alianzas estratégicas para combatir las amenazas:
La asociación de infraestructura crítica del sector de servicios financieros incluye una variedad de partes interesadas además de la FSSCC y el FBIIC:
\newline
\newline
• Sector privado: FSSCC, Centro de Análisis e Intercambio de Información de Servicios Financieros (FSISAC), empresas individuales, asociaciones comerciales, coaliciones regionales, proveedores de servicios de seguridad, proveedores de servicios tecnológicos y socios industriales de otros sectores;
\newline
\newline
• Poder Ejecutivo: Departamento del Tesoro, DHS (incluido el Servicio Secreto de los Estados Unidos), Departamento de Justicia de los Estados Unidos (incluida la Oficina Federal de Investigaciones), Departamento de Defensa de los Estados Unidos y otros departamentos y agencias;
\newline
\newline
• Reguladores Financieros: Agencias FBIIC, que incluye reguladores bancarios y de cooperativas de crédito; reguladores de valores; organizaciones de autorregulación; y reguladores estatales;
\newline
\newline
• Socios estatales, locales, tribales y territoriales; 
\newline
\newline
• Internacional: fuera de EEUU instituciones financieras y proveedores de servicios, reguladores fuera de los EEUU y socios gubernamentales de seguridad nacional, comunidad de inteligencia y aplicación de la ley fuera de los EEUU.
 \newline
\newline
Para mejorar su seguridad y resiliencia, el sector trabaja para avanzar en cuatro objetivos principales, que brindan un marco para identificar y priorizar programas e iniciativas de colaboración, especialmente entre el FBIIC y la FSSCC:
 \newline
\newline
1. Implementar y mantener rutinas estructuradas para compartir información oportuna y procesable relacionada con la seguridad cibernética y las amenazas y vulnerabilidades físicas entre empresas, entre sectores de la industria y entre el sector privado y el gobierno.
\newline
\newline
2. Mejorar las capacidades de gestión de riesgos y la postura de seguridad de las empresas en todo el sector de servicios financieros y los proveedores de servicios en los que confían fomentando el desarrollo y uso de enfoques comunes y mejores prácticas.
\newline
\newline
3. Colaborar con las comunidades de seguridad nacional, aplicación de la ley e inteligencia; autoridades reguladoras financieras; otros sectores de la industria; y socios internacionales para responder y recuperarse de incidentes significativos.
\newline
\newline
4. Discutir políticas e iniciativas regulatorias que promuevan la seguridad de la infraestructura y las prioridades de resiliencia a través de una sólida coordinación entre el gobierno y la industria.
\newline
\section{Marco internacional}
Si bien es cierto que la guerra fría terminó el 26 de diciembre de 1991, nunca
cesó el miedo del mundo occidental de un posible ascenso del marco soviético. Sin embargo, el hecho de que China desplazara a EEUU como potencia mundial, fue totalmente inesperado.
\newline
\newline
Xi Jinpin lleva al rededor de 10 años a cargo de China, y en este tiempo ha reestructurado el país y cambiado su política económica. 
\newline
\newline
Sus maneras expansionistas han afectado el marco geopolítico mundial, incluyendo Latinoamérica. 
\newline
\newline
La Nueva Ruta de la Seda se expande no solo por Europa, sino también por Latinoamérica.
China tomó conciencia de el desinterés de la zona latinoamericana por parte de los Estados Unidos y comenzó un plan de generar infraestructuras tanto estructurales como físicas y digitales para el mejoramiento del intercambio cultural, económico y financiero.
\newline
\newline
EEUU por su parte comenzó un plan llamado B3W, cuyo fin es contrarrestar el efecto que va a tener esta política expansionista China, cuyo origen data del 2019 en la administración de Donald Trump con el Blue Dot Network, renovado por Joe Biden en junio de 2021 y se extiende hasta el 2035. El plan B3W, en el que se han comprometido Japón, Reino Unido, Canadá, Francia, Alemania e Italia, propone canalizar colectivamente billones de dólares de inversión pública y privada para nuevas carreteras, ferrocarriles, puertos y redes de comunicación en países de ingresos bajos y medios. Un problema frente a su credibilidad es que no tiene dinero propio por el momento.
\newline
\newline
Por el contrario, la Nueva Ruta de la Seda está respaldada por las arcas repletas de reservas de Pekín, lo que lleva a algunos expertos a tachar el B3W de ser solo una promesa vaga que hace falta cumplir. China ya ha financiado trenes, carreteras y puertos, y las empresas de construcción chinas han obtenido contratos lucrativos para conectar puertos y ciudades, financiados por préstamos de bancos chinos.
\newline
\newline
Estas políticas buscan invertir en infraestructura en nuestro continente con el fin de generar lazos y verse fortalecidos frente a los cambios que puedan darse en el mundo ante todo macroeconómicamente. 
\newline
\newline
La Guerra Rusia-Ucrania ha costado a China una gran cantidad de dinero, no solo por las alzas en los precios de algunos commodities, sino porque ha tenido que salir del mercado europeo (compraba maíz a Ucrania) e incursionar en otros mercados. 
\newline
\newline
Esto es un ejemplo de por qué la resiliencia de una economía y de su sector financiero son importantes. Inminentemente un conflicto ocurriría, antes o temprano, sea por Rusia o por cualquier otro país, una tensión lleva a una invasión que se desata en una guerra y los mercados se ven afectados. La infraestructura de todo un país se ve comprometida, Alemania y su dependencia en el gaseoducto con Rusia es el ejemplo perfecto, un país sólido con una estructura bien armada se ven comprometidos por su fuerte dependencia con un recurso específico proveniente mayormente por un único proveedor.
\newline
\newline
Pero, ¿Cómo se ve afectada Latinoamérica? 
Inminentemente hay una competencia entre dos grandes potencias por quien se hace del dominio económico de este sector. Ruta de la seda digital, B3W y nueva ruta de la seda no son mas que políticas para favorecer tanto a China como EEUU y hacer mas resiliente su infraestructura crítica. Depende de nosotros saber aprovechar esto y generar lazos no solo con un país, sino dentro de nuestra región y con el mundo; mejorando comunicación, intercambio e infraestructura física, así haciendo mas resiliente a nuestro sector financiero, ya que, ante cualquier distención, podremos estar preparados. 
\newline
\newline
Si bien estas ayudas no son gratis, un endeudamiento con estas potencias no es malo si es inteligentemente manejado para no ser socio exclusivo con una de ellas, y en cambio es utilizado para hacer crecer nuestras economías, volverlas mas competitivas y abiertas, y sobre todo más resistentes frente a cambios que, en este mundo completamente globalizado no serán excepcionales.




\section{Sector Financiero: Argentina}
Los vaivenes en la economía son algo característico de nuestro país. Como es sabido, el perfil político define el modelo económico y es por ello que se adopta tal o cual medida, generalmente macroeconómica.

En la actualidad, la Argentina se encuentra en un momento de cierta inestabilidad económica y financiera. Nuestro modelo es caracterizado por el intervencionismo estatal y una férrea protección de la industria nacional. Políticas las cuales han generado que seamos un país industrialmente poco competitivo a nivel mundial, sin embargo siendo fuertemente dependientes de insumos extranjeros. 

Un mercado como el argentino es muy sensible a los cambios macroeconómicos mundiales. Un alza en el precio del dólar, una baja en el precio de la tonelada de soja, una guerra en Europa del este, o cualquier conflicto, por mas chico que sea, termina repercutiendo fuertemente en nuestro sistema financiero. 

Además, dada la falta de seguridad jurídica y de previsibilidad en los negocios, se genera que capitales extranjeros se establezcan en otros países como Uruguay, Chile o Brasil y aquellos que se encuentran en la Argentina emigren.

Como consecuencia de todo esto, no somos capaces de adaptarnos rápidamente y nos vemos perjudicados ya que perdemos oportunidad de introducirnos a nuevos mercados, formar mejores y más prósperos lazos, y de poder afianzarnos como país, dejar atrás los problemas económicos y así en un futuro, los problemas sociales a gran escala. 


\section{Reflexión}
El sector financiero es el motor de las economías y del desarrollo, da una idea de la seriedad de un país y es fundamental para el crecimiento, sin un sector financiero sólido es difícil que el país prospere y mucho menos que se posicione como potencia. Como bien hemos dicho es una de las infraestructuras críticas más importantes debido a la interdependencia de muchas otras infraestructuras críticas con éste. La vida de las personas, las empresas y el gobierno tienen sus cimientos basados en esto, y de caer, se generaría una gran crisis. Esto se puede observar con la gran sensibilidad de los mercados mundiales a una crisis en cualquier sistema medianamente importante. ¿Estamos preparados estratégicamente para afrontar una crisis financiera grande? ¿Cómo debemos responder como país? ¿Cómo podemos aprovechar una hipotética situación de esa índole para ayudar al sistema mundial y a la vez salir mejor parados como país?

\section{Fuentes}

\href{https://atcee.es/consecuencias-de-la-crisis-del-sistema-financiero/}{ https://atcee.es/consecuencias-de-la-crisis-del-sistema-financiero/ }
\newline
\newline
\href{https://www.cisa.gov/financial-services-sector}{https://www.cisa.gov/financial-services-sector}
\newline
\newline
\href{https://www.lisainstitute.com/blogs/blog/infraestructuras-criticas}{ https://www.lisainstitute.com/blogs/blog/infraestructuras-criticas }
\newline
\newline
\href{https://aecconsultoras.com/noticias-sectoriales/los-bancos-en-jaque-por-el-covid-19/}{https://aecconsultoras.com/noticias-sectoriales/los-bancos-en-jaque-por-el-covid-19}
\newline
\newline
\href{https://www.cisa.gov/sites/default/files/publications/nipp-ssp-financial-services-2015-508.pdf
}{https://www.cisa.gov/sites/default/files/publications/nipp-ssp-financial-services-2015-508.pdf}
\newline
\newline
\href{https://elordenmundial.com/que-es-la-nueva-ruta-de-la-seda-china/}{https://elordenmundial.com/que-es-la-nueva-ruta-de-la-seda-china/}
\newline
\newline
\href{https://www.esglobal.org/b3w-la-apuesta-de-biden-contra-la-ruta-de-la-seda/}{https://www.esglobal.org/b3w-la-apuesta-de-biden-contra-la-ruta-de-la-seda/}
\newline
\newline
\href{https://www.bbc.com/mundo/noticias-america-latina-48071584}{https://www.bbc.com/mundo/noticias-america-latina-48071584}
\newline
\newline
\href{https://aduba.org.ar/wp-content/uploads/2016/07/SISTEMA-FINANCIERO-ARGENTINO-2014-1-4.pdf}{https://aduba.org.ar/wp-content/uploads/2016/07/SISTEMA-FINANCIERO-ARGENTINO-2014-1-4.pdf}



\printindex
\end{document}


